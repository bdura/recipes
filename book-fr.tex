\documentclass[9pt]{extarticle}

\usepackage{fancyhdr}

\usepackage{multicol}
\usepackage{enumitem}

\usepackage[
    a5paper,
    margin=2cm,
    top=3cm,
    bottom=3cm
]{geometry}

\usepackage{xcolor}
\usepackage{graphicx}

% Configuration des paragraphes : pas d'alinéa, saut de ligne
\usepackage{parskip}

\usepackage{titlesec}

% \titleformat*{\section}{\LARGE\bfseries}
\titleformat*{\subsection}{\LARGE\bfseries}

% \usepackage{etoc}

\raggedcolumns

\setlength{\multicolsep}{0pt}
\setlength{\columnseprule}{0pt}

\makeatletter

\newif\if@mainmatter \@mainmattertrue

\newcommand\frontmatter{%
    \cleardoublepage
  \@mainmatterfalse
  \pagenumbering{roman}}
\newcommand\mainmatter{%
    \cleardoublepage
  \@mainmattertrue
  \pagenumbering{arabic}}
\makeatother

\newcommand{\nnsection}[1]{
  \section*{#1}
  \addcontentsline{toc}{section}{#1}
}
\newcommand{\nnsubsection}[1]{
  \subsection*{#1}
  \addcontentsline{toc}{subsection}{#1}
}

\definecolor{vegcolor}{rgb}{0,0.5,0.2}
\definecolor{frzcolor}{rgb}{0,0,1}

\newcommand{\serves}[1]{%
    \rhead{#1}
}

\newcommand{\preptime}[2][Preparation]{%
    \lhead{#1: #2}%
%    \lfoot{#1: #2}%
}
\newcommand{\cooktime}[2][Cook time]{%
    \rhead{#1: #2}%
%    \rfoot{#1: #2}%
}

\pagestyle{fancy}


\begin{document}

\frontmatter

\tableofcontents

\mainmatter

\cleardoublepage

\nnsection{Salé}

\cleardoublepage

  \clearpage

\nnsubsection{Pâtes fraiches}

\serves{5 portions}
\preptime{20 minutes}
\cooktime{10 minutes}

% \vspace{.2cm}

\begin{center}
    \parbox{0.8\linewidth}{
        \raggedright\itshape

        
    }
\end{center}

% \vspace{.2cm}

{\large \bf Ingredients}

\vspace{.2cm}

\begin{multicols}{2}

    \begin{itemize}[label={},left=0pt,parsep=10pt]
    \item 400g de farine
    \item 4 oeufs
    \item 15ml d'eau
    \item une pincée de sel
\end{itemize}

\end{multicols}



\vspace{.4cm}

{\large \bf Instructions}

\begin{enumerate}[left=0pt,noitemsep]
    \item Mélanger les ingrédients et pétrir la pâte pendant quelques minutes.
    \item Laisser reposer au moins trente minutes au réfrigérateur.
    \item Aplatir et découper la pâte.
\end{enumerate}  \clearpage

\nnsubsection{Granola}

\serves{600g de granola}
\preptime{10 minutes}
\cooktime{20 à 25 minutes}

% \vspace{.2cm}

\begin{center}
    \parbox{0.8\linewidth}{
        \raggedright\itshape

        
    }
\end{center}

% \vspace{.2cm}

{\large \bf Ingredients}

\vspace{.2cm}

\begin{multicols}{2}

    \begin{itemize}[label={},left=0pt,parsep=10pt]
    \item 430g de grands flocons d'avoine
    \item ~100g de graines ou noix
    \item 180g de sirop d'érable (ou de miel)
    \item 120g de beurre de coco (ou autre huile végétale)
    \item une cuillère à café de sel
    \item une pincée d'extrait de vanille
\end{itemize}

\end{multicols}



\vspace{.4cm}

{\large \bf Instructions}

\begin{enumerate}[left=0pt,noitemsep]
    \item Préchauffer le four à 180$^\circ$C.
    \item Mélanger les flocons, les noix, la canelle, l'extrait de vanille et le sel dans un grand bol.
    \item Ajouter le sirop d'érable/miel et le beurre de coco fondu puis mélanger jusqu'à ce que les flocons soient tous bien imprégnés.
    \item Disposer le mélange sur une plaque de cuisson recouverte d'une feuille de papier sulfurisé. Veiller à ce que le mélange forme une épaisseur uniforme.
    \item Enfourner entre 20 et 25 minutes à 180$^\circ$C. À mi-cuisson, mélanger les flocons pour une cuisson plus uniforme. Pour obtenir des gros morceaux après la cuisson, bien aplatir avec une cuillère en bois.
    \item Laisser reposer 45 minutes à la sortie du four afin que le mélange refroidisse tranquillement. C'est à ce moment-là que les flocons se lient entre eux.
\end{enumerate}  \clearpage

\nnsubsection{Pain au levain}

\serves{1 pain}
\preptime{20 minutes}
\cooktime{45 minutes}

% \vspace{.2cm}

\begin{center}
    \parbox{0.8\linewidth}{
        \raggedright\itshape

        Un bon pain au levain...
    }
\end{center}

% \vspace{.2cm}

{\large \bf Ingredients}

\vspace{.2cm}

\begin{multicols}{2}

    \begin{itemize}[label={},left=0pt,parsep=10pt]
    \item 500g de farine blanche
    \item 150g de levain
    \item 330ml d'eau
    \item 10g de sel
\end{itemize}

\end{multicols}



\vspace{.4cm}

{\large \bf Instructions}

\textit{Pointage}
\begin{enumerate}[left=0pt,noitemsep]
    \item Mélanger les ingrédients.
    \item Pétrir pendant 10 minutes. Si vous avez un robot pétrisseur, faites-le fonctionner à vitesse minimale.
    \item Sur un plan de travail fariné.
    \item Laisser reposer six à douze heures sous un torchon humide.
\end{enumerate}

\textit{Façonnage}
\begin{enumerate}[left=0pt,noitemsep]
    \item Sortir le pâton délicatement sur un plan de travail fariné.
    \item Replier le pâton sur lui-même en tâchant de ne pas perturber les bulles.
    \item Remettre le pâton sur un torchon propre, sec et fariné, la “couture” vers le haut. recouvrir avec le reste du torchon.
    \item Laisser reposer une heure à une heure et demie.
\end{enumerate}

\textit{Cuisson}
\begin{enumerate}[left=0pt,noitemsep]
    \item Préchauffer le four à 240$^\circ$C. Si vous avez une cocotte en fonte, mettez-la à chauffer aussi.
    \item Sur un plan de travail fariné, sortir délicatement le pâton de son torchon, coutûre vers le bas.
    \item Avec un couteau très bien aiguisé, pratiquer une ou plusieurs incision.
    \item Enfourner.
\end{enumerate}  \clearpage

\nnsubsection{Levain}

\serves{1 levain}
\preptime{5 minutes par jour}
\cooktime{1 semaine}

% \vspace{.2cm}

\begin{center}
    \parbox{0.8\linewidth}{
        \raggedright\itshape

        
    }
\end{center}

% \vspace{.2cm}

{\large \bf Ingredients}

\vspace{.2cm}

\begin{multicols}{2}

    \textit{Chaque jour}
\begin{itemize}[label={},left=0pt,parsep=10pt]
    \item 50g de farine de seigle
    \item 50ml d'eau
\end{itemize}

\end{multicols}



\vspace{.4cm}

{\large \bf Instructions}

\textit{Premier jour}
\begin{enumerate}[left=0pt,noitemsep]
    \item Mélanger l'eau et la farine de seigle. On peut utiliser un autre type de farine mais pour une raison qui m'échappe le seigle semble fonctionner beaucoup mieux...
    \item Laisser reposer 24h dans un bocal propre, fermé par un tissus lâche (juste pour éviter que des insectes y rentrent) et à température ambiante.
\end{enumerate}

\textit{Les jours suivants}
\begin{enumerate}[left=0pt,noitemsep]
    \item Retirer l'excès de levain pour que le bocal ne soit pas rempli à plus de la moitié.
    \item Répéter l'opération chaque jour tant que le levain n'est pas colonisé. Une fois que le levain est lancé, il double de volume après six heures, donc c'est facile à détecter !
\end{enumerate}

\textit{Prendre soin de son levain}
\begin{enumerate}[left=0pt,noitemsep]
    \item Nourrir (rafraichir) le levain au moins une fois par semaine (ou toutes les deux semaines s'il est au réfrigérateur).
    \item Il faut que le levain soit en pleine forme pour faire du pain : pensez à le rafraichir une ou deux fois avant de vous lancer dans la confection d'un pain, surtout si vous l'aviez mis en hibernation...
\end{enumerate}
\end{document}